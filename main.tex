\documentclass[conference]{IEEEtran}
\IEEEoverridecommandlockouts
% The preceding line is only needed to identify funding in the first footnote. If that is unneeded, please comment it out.
\usepackage{cite}
\usepackage{amsmath,amssymb,amsfonts}
\usepackage{algorithmic}
\usepackage{graphicx}
\usepackage{textcomp}
\usepackage[colorlinks,bookmarksopen,bookmarksnumbered,citecolor=red,urlcolor=red]{hyperref}
\newcommand{\fp}[1]{ {\textcolor{green}    { **Fangping:  #1 }}}
\newcommand{\bm}[1]{ {\textcolor{blue}    { **Barry:  #1 }}}
\newcommand{\kim}[1]{ {\textcolor{cyan}    { **Kim:  #1 }}}
\newcommand{\shervin}[1]{ {\textcolor{black}    { **Shervin:  #1 }}}
\newcommand{\ketan}[1]{ {\textcolor{red}    { **Ketan:  #1 }}}

\def\BibTeX{{\rm B\kern-.05em{\sc i\kern-.025em b}\kern-.08em
    T\kern-.1667em\lower.7ex\hbox{E}\kern-.125emX}}
\begin{document}

\title{Paper Title*\\
{\footnotesize \textsuperscript{*}Note: Sub-titles are not captured in Xplore and
should not be used}
\thanks{Identify applicable funding agency here. If none, delete this.}
}

\author{\IEEEauthorblockN{1\textsuperscript{st} Given Name Surname}
\IEEEauthorblockA{\textit{dept. name of organization (of Aff.)} \\
\textit{name of organization (of Aff.)}\\
City, Country \\
email address}
\and
\IEEEauthorblockN{2\textsuperscript{nd} Given Name Surname}
\IEEEauthorblockA{\textit{dept. name of organization (of Aff.)} \\
\textit{name of organization (of Aff.)}\\
City, Country \\
email address}
\and
\IEEEauthorblockN{3\textsuperscript{rd} Given Name Surname}
\IEEEauthorblockA{\textit{dept. name of organization (of Aff.)} \\
\textit{name of organization (of Aff.)}\\
City, Country \\
email address}
\and
\IEEEauthorblockN{4\textsuperscript{th} Given Name Surname}
\IEEEauthorblockA{\textit{dept. name of organization (of Aff.)} \\
\textit{name of organization (of Aff.)}\\
City, Country \\
email address}
\and
\IEEEauthorblockN{5\textsuperscript{th} Given Name Surname}
\IEEEauthorblockA{\textit{dept. name of organization (of Aff.)} \\
\textit{name of organization (of Aff.)}\\
City, Country \\
email address}
\and
\IEEEauthorblockN{6\textsuperscript{th} Given Name Surname}
\IEEEauthorblockA{\textit{dept. name of organization (of Aff.)} \\
\textit{name of organization (of Aff.)}\\
City, Country \\
email address}
}

\maketitle

\begin{abstract}
This document is a model and instructions for \LaTeX.
This and the IEEEtran.cls file define the components of your paper [title, text, heads, etc.]. *CRITICAL: Do Not Use Symbols, Special Characters, Footnotes, 
or Math in Paper Title or Abstract.
\end{abstract}

\begin{IEEEkeywords}
Storage systems, ZFS, HPC
\end{IEEEkeywords}



\section{Introduction}

\fp{ Test Comment } \bm{ Test Comment } \ketan{ Test Comment }  \kim{ Test Comment }


The Center for Research Computing (CRC) is the University's closest thing to research computing.
While CRC maintains the University's largest cluster, the
funding model has resulted in a highly heterogeneous architecture that is essentially a {\it Frankenstein} of
many smaller clusters. In the past year, CRC tried to meet the following requirements:

\begin{itemize}

\item Merge any small portion of the cluster around the university. 

\item Significant investment in hardware, in terms of compute, storage and networking from the Provost Office had been made. 

\item The CRC grows in terms of personnel. Personnel are dedicated system administrators. These administrators are Ph.D. level scientists and are in charge of system admin, user admin and daily
technical support. They are also domain experts with high performance computing expertise with significant computational research experience. Research computing is the blending of traditional science with cutting-edge computational expertise. As such an excellent scientist with no computational skills will not
be successful in this space. Likewise, an excellent computer scientist/programmer will not be successful in this space. However, domain experts with both the science and computing background,
capable of interfacing with both traditional scientist and computer scientists are the glue required
for successful collaborations in research computing. Such a model is the status quo at all major
national labs and national computing centers.

\end{itemize}


\shervin{
The University of Pittsburgh is best served by servicing the missing middle. The missing middle is the void
between lab level and department level computing resources which order $O(10^1)-O(10^2)$ processors and national resources which order $O(10^5)-O(10^6)$ processors. In the same way that the
University's mission is to develop and prepare graduates for the workforce, research computing
at the University should develop and prepare computational researcher's for the utilization of
facilities at the largest scale (national labs).}

\begin{itemize}
\item Outreach
\item Training (example is Barmeda's workshop, Bio-Stat course for undergraduate students.
\item Workforce developement
\item Hacketown
\end{itemize}

\subsection{Engagement}
One of the main duties of Consultants at CRC is meeting with	researchers	in order to better	understand their computational needs and to help them select appropriate resources and approaches. These engagements are important for accelerating the adoption of ACI resources in support of research endeavors. At Pit, we are also engaged in face-to-face meeting with researchers. For that, regular office hours are assigned so that students can meet with CRC team.

\subsection{Effort to be done}
Compute is
cheap, storage is not, having all storage backed up is even more expensive. The University needs
to recognize that this requires significant investment, especially in regards to big data. Moreover,
an upgrade path and budget to replace and extend these resources needs to be in place.

\section{Conclusions}

Conclusions go here.

%ACKNOWLEDGMENTS are optional
\section{Acknowledgments}
Acknowledge CRC.
%
% The following two commands are all you need in the
% initial runs of your .tex file to
\bibliographystyle{abbrv}
\bibliography{ref} 
\end{document}

====== IGNORE ========
Comments and brainstorming go here!

High Performance Computing (HPC) and, in general, Parallel and Distributed Computing (PDC) has become pervasive, from supercomputers and server farms containing multicore CPUs and GPUs, to individual PCs, laptops, and mobile devices. Even casual users of computers now depend on parallel processing. Therefore it is important for every computer user (and especially every programmer) to understand how parallelism and distributed computing affect problem solving. It is essential for educators to impart a range of PDC and HPC knowledge and skills at multiple levels within the educational fabric woven by Computer Science (CS), Computer Engineering (CE), and related computational curricula including data science. Companies and laboratories need people with these skills, and, as a result, they are finding that they must now engage in extensive on-the-job training. Nevertheless, rapid changes in hardware platforms, languages, and programming environments increasingly challenge educators to decide what to teach and how to teach it, in order to prepare students for careers that are increasingly likely to involve PDC and HPC.

This workshop invites unpublished manuscripts from academia, industry, and government laboratories on topics pertaining to the needs and approaches for augmenting undergraduate and graduate education in Computer Science and Engineering, Computational Science, and computational courses for both STEM and business disciplines with PDC and HPC concepts.  We also encourage papers on large-scale data science.

The workshop is particularly dedicated to bringing together stakeholders from industry (both hardware vendors and employers), government labs, and academia in the context of SC-17.  The goal is for each to hear the challenges faced by others, to learn about various approaches to addressing these challenges, and to have opportunities to exchange ideas and solutions. In addition to contributed talks, this workshop may feature invited talks on opportunities for collaboration, resource sharing, educator training, internships, and other means of increasing cross-fertilization between industry, government, and academia.

Topics of interest include, but are not limited to:

1. Pedagogical issues in incorporating PDC and HPC in undergraduate and graduate education, especially in core courses
2. Novel ways of teaching PDC and HPC topics
3. Data Science and Big Data aspects of teaching HPC/PDC including early experience with data science degree programs.
4. Experience with incorporating PDC and HPC topics into core CS/CE courses and in domain Computational Science and Engineering courses
5. Pedagogical tools, programming environments, infrastructures, languages, and projects for PDC and HPC
6. Employers' experiences with and expectation of the level of PDC and HPC proficiency among new graduates
7. Education resources based on higher-level programming languages, models, and environments such as PGAS, X10, Chapel, Haskell, Python, Cilk, CUDA, OpenCL, OpenACC, and Hadoop  
8. Parallel and distributed models of programming and computation suitable for teaching, learning, and workforce development.
9. Projects or units that introduce students to concepts relevant to Internet of Things, networking, or other topics in mobile devices or sensor networks.
10. Issues and experiences addressing the gender gap in computing and broadening participation of underrepresented groups.
